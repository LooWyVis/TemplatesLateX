\documentclass[11pt]{exam}

\usepackage[a4paper,margin={.1\paperheight,.1\paperwidth},marginratio=1:1]{geometry}
\usepackage[utf8]{inputenc}   %Pour MAC 
\usepackage[french]{babel}
\usepackage[T1]{fontenc}
\usepackage{amsmath}  % Ajout du package pour les mathématiques
\usepackage{amsfonts} % Facultatif, mais utile pour les symboles supplémentaires
\usepackage{siunitx}  % Pour les unités de mesure


\newcommand{\doublebar}[0]{
    \noindent\rule{\textwidth}{3pt}
    \rule[3mm]{\textwidth}{0.5pt}
}

\newcounter{compteurpartie}
\newcounter{compteurexercice}

\newcommand{\partie}[1]{
    \stepcounter{compteurpartie}
    \section*{Partie \arabic{compteurpartie} : \hspace{1pt} #1}
    \setcounter{compteurexercice}{0} 
}

\newcommand{\exercice}[2]{
    \stepcounter{compteurexercice}
    \section*{Exercice \arabic{compteurexercice} \hrulefill \hspace{1pt} #1}
}


\begin{document}

% Header
\noindent Lycée Madeleine Michelis \hfill Mathématiques

\vspace{2em}
\begin{center}
    \textbf{Équations du Second Degré}

    \vspace{1em}
    Exercices
\end{center}
    
\doublebar

% Content
\partie{Résolution d'équations du second degré}

\exercice{Résolution d'une équation simple}

Résoudre l'équation suivante :
\[
x^2 - 5x + 6 = 0
\]


\exercice{Discriminant}

Pour l'équation \( 2x^2 - 4x - 3 = 0 \), calculer le discriminant et en déduire les solutions de l'équation.


\partie{Applications des équations du second degré}

\exercice{Problème appliqué}

Un objet est projeté verticalement vers le haut avec une vitesse initiale de \( 20 \, \text{m/s} \). Sa hauteur \( h(t) \) en fonction du temps est donnée par l'équation \( h(t) = -5t^2 + 20t + 2 \). À quel moment l'objet atteint-il la hauteur maximale ?


\exercice{Facteur commun}

Résoudre l'équation suivante en factorisant :
\[
3x^2 + 12x + 12 = 0
\]


\end{document}
