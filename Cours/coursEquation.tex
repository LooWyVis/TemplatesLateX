\documentclass[11pt]{exam}

\usepackage[a4paper,margin={.1\paperheight,.1\paperwidth},marginratio=1:1]{geometry}
%\usepackage[latin1]{inputenc}  %Pour Windows
\usepackage[utf8]{inputenc}   %Pour MAC 
\usepackage[french]{babel}
\usepackage[T1]{fontenc}
\usepackage{amsmath}  % Ajout pour les mathématiques

\newcommand{\doublebar}[0]{
    \noindent\rule{\textwidth}{3pt}
    \rule[3mm]{\textwidth}{0.5pt}
}

\newcounter{compteurpartie}
\newcounter{compteurexercice}

\newcommand{\partie}[1]{
    \stepcounter{compteurpartie}
    \section*{Partie \arabic{compteurpartie} : \hspace{1pt} #1}
    \setcounter{compteurexercice}{0} 
}

\newcommand{\exercice}[2]{
    \stepcounter{compteurexercice}
    \section*{\arabic{compteurpartie}.\arabic{compteurexercice} \hrulefill \hspace{1pt} #1}
}

\begin{document}

% Header
\noindent Lycée Madeleine Michelis \hfill Mathématiques

\vspace{2em}
\begin{center}
    \textbf{Équations du Second Degré}

    \vspace{1em}
    Cours
\end{center}
    
\doublebar


% Content
\partie{Méthode de Résolution d'une Équation du Second Degré}

\exercice{Forme générale}

Une équation du second degré se présente sous la forme générale suivante :
\[
ax^2 + bx + c = 0
\]
où \(a\), \(b\) et \(c\) sont des coefficients réels, et \(a \neq 0\).

\exercice{Le discriminant}

Pour résoudre une équation du second degré, on commence par calculer le discriminant \(\Delta\) donné par la formule :
\[
\Delta = b^2 - 4ac
\]
Selon la valeur de \(\Delta\), il existe trois cas possibles :

\begin{itemize}
    \item Si \(\Delta > 0\), l'équation a deux solutions réelles distinctes.
    \item Si \(\Delta = 0\), l'équation a une solution réelle double.
    \item Si \(\Delta < 0\), l'équation n'a pas de solution réelle, mais deux solutions complexes.
\end{itemize}

\exercice{Les solutions}

Les solutions de l'équation dépendent du signe du discriminant \(\Delta\).

\begin{enumerate}
    \item \textbf{Cas 1 : \(\Delta > 0\)} \\
    L'équation a deux solutions réelles distinctes données par :
    \[
    x_1 = \frac{-b - \sqrt{\Delta}}{2a}, \quad x_2 = \frac{-b + \sqrt{\Delta}}{2a}
    \]
    
    \item \textbf{Cas 2 : \(\Delta = 0\)} \\
    L'équation a une solution réelle double :
    \[
    x = \frac{-b}{2a}
    \]
    
    \item \textbf{Cas 3 : \(\Delta < 0\)} \\
    L'équation n'a pas de solution réelle. Les solutions sont complexes :
    \[
    x_1 = \frac{-b - i\sqrt{|\Delta|}}{2a}, \quad x_2 = \frac{-b + i\sqrt{|\Delta|}}{2a}
    \]
\end{enumerate}

\exercice{Exemple}

Considérons l'équation suivante :
\[
2x^2 - 4x - 6 = 0
\]
Nous avons ici \(a = 2\), \(b = -4\), et \(c = -6\). Calculons le discriminant :
\[
\Delta = (-4)^2 - 4 \times 2 \times (-6) = 16 + 48 = 64
\]
Comme \(\Delta > 0\), l'équation a deux solutions réelles distinctes :
\[
x_1 = \frac{-(-4) - \sqrt{64}}{2 \times 2} = \frac{4 - 8}{4} = -1
\]
\[
x_2 = \frac{-(-4) + \sqrt{64}}{2 \times 2} = \frac{4 + 8}{4} = 3
\]
Ainsi, les solutions de l'équation sont \(x_1 = -1\) et \(x_2 = 3\).

\end{document}
