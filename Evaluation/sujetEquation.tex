\documentclass[11pt]{exam}

\usepackage[a4paper,margin={.1\paperheight,.1\paperwidth},marginratio=1:1]{geometry}
\usepackage[utf8]{inputenc}
\usepackage[french]{babel}
\usepackage[T1]{fontenc}
\usepackage{amssymb} % Ajout du package pour les symboles mathématiques
\pagestyle{headandfoot}

% Commande pour dessiner une double ligne
\newcommand{\doublebar}[0]{
    \noindent\rule{\textwidth}{3pt} % Ligne épaisse en haut
    \rule[3mm]{\textwidth}{0.5pt}    % Ligne fine en dessous
}

% Compteurs pour les parties et les exercices
\newcounter{compteurpartie}
\newcounter{compteurexercice}

% Commande pour créer une nouvelle partie
\newcommand{\partie}[1]{
    \stepcounter{compteurpartie}
    \section*{Partie \arabic{compteurpartie} : \hspace{1pt} #1}
    \setcounter{compteurexercice}{0} % Réinitialiser le compteur d'exercices
}

% Commande pour créer un nouvel exercice
\newcommand{\exercice}[2]{
    \stepcounter{compteurexercice}
    \section*{Exercice \arabic{compteurexercice} \tiny{/#2 points} \hrulefill \hspace{1pt} #1}
}

% Début du document
\begin{document}

\firstpageheader{Nom : \\ Prénom :}
  {}
  {classe : \texttt{Terminale} \\ Date : \today}
\firstpagefooter{}{Page~\thepage}{}

\vspace{2em}
\begin{center}
    \textbf{Équations du second degré}

    \vspace{1em}
    Évaluation
\end{center}
    
\doublebar

% Contenu
\partie{Résolution d'équations du second degré}

\exercice{}{5}
Résoudre l'équation \(2x^2 - 4x - 6 = 0\) en utilisant la formule quadratique.

\exercice{}{5}
Trouver les racines de l'équation \(x^2 + 3x + 2 = 0\) par factorisation.

\partie{Analyse de fonctions quadratiques}

\exercice{}{10}
Donner le sommet de la parabole de l'équation \(y = -x^2 + 4x - 3\).

\exercice{}{5}
Déterminer le signe de la fonction \(f(x) = x^2 - 5x + 6\) sur \(\mathbb{R}\).

\partie{Applications des équations du second degré}

\exercice{}{10}
Un projectile est lancé verticalement vers le haut. Sa hauteur en fonction du temps est donnée par l'équation \(h(t) = -5t^2 + 20t + 10\). Déterminer le temps au bout duquel le projectile atteint sa hauteur maximale.

\exercice{}{10}
Un rectangle a une aire de 60 m². Si la longueur est 2 m plus grande que la largeur, établir et résoudre l'équation du second degré pour trouver les dimensions du rectangle.

\end{document}
