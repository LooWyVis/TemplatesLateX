% !TEX encoding = UTF-8 Unicode
% !TEX TS-program = lualatex
\documentclass[12pt,%
addpoints,%answers%
]{exam}

\usepackage{unicode-math}
\usepackage[frenchb]{babel}
\pagestyle{headandfoot}

\firstpageheader{Nom : \\ Prénom :}
  {}
  {classe : \texttt{www.cuk.ch} \\ Date : \today}
\firstpagefooter{}{Page~\thepage}{}
\hsword{Cote:}
\begin{document}
\begin{center}
  {\huge\bfseries\sffamily Contrôle pour rire}
  \par\bigskip
  \gradetable[h]
\end{center}
\begin{questions}
  \question[1\half] Démontrer que \( (a+b)^2 = a^2 + 2ab + b^2 \).
  \begin{solution}[5\bigskipamount]
    \[
      (a+b)^2 = (a+b)\times(a+b) = a^2 + ab + ba + b^2 = a^2 + 2ab + b^2.
    \]
  \end{solution}
  \question[1\half] Quelle était la couleur du cheval blanc d'Henri~IV ?
  \begin{solution}[3\bigskipamount]
    Noir avant d'être recouvert de farine.
  \end{solution}
  
  \question[2] La bataille de Marignan est une des plus célèbres de l'histoire de France.
  
  \begin{parts}
  
    \part Quand a-t-elle eu lieu ?
  
    \begin{solution}[3\bigskipamount]
      En 1515.
    \end{solution}
    \part Comment s'appelait le roi de France vainqueur de cette bataille ?
    \begin{solution}[3\bigskipamount]
      François~1\ier.
    \end{solution}
  \end{parts}
\end{questions}
\end{document}