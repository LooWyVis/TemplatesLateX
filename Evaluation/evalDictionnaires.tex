\documentclass[11pt]{exam}


% Packages
\usepackage[a4paper,margin={.1\paperheight,.1\paperwidth},marginratio=1:1]{geometry}
%\usepackage[latin1]{inputenc}  %Pour Windows
\usepackage[utf8]{inputenc}   %Pour MAC 
\usepackage[french]{babel}
\usepackage[T1]{fontenc}
\usepackage{pifont}
\usepackage{amssymb}
\usepackage{ifthen}
\usepackage{tcolorbox}
\tcbuselibrary{minted,breakable,xparse,skins}
\DeclareTCBListing{mintedbox}{O{}m!O{}}{%
  breakable=true,
  listing engine=minted,
  listing only,
  minted language=#2,
  minted style=default,
  minted options={%
    linenos,
    gobble=0,
    breaklines=true,
    breakafter=,,
    fontsize=\small,
    numbersep=8pt,
    #1},
  boxsep=0pt,
  left skip=0pt,
  right skip=0pt,
  left=25pt,
  right=0pt,
  top=3pt,
  bottom=3pt,
  arc=5pt,
  leftrule=0pt,
  rightrule=0pt,
  bottomrule=2pt,
  toprule=2pt,
  colback=white,
  colframe=orange!70,
  enhanced,
  overlay={%
    \begin{tcbclipinterior}
    \fill[orange!20!white] (frame.south west) rectangle ([xshift=20pt]frame.north west);
    \end{tcbclipinterior}},
  #3}


\pagestyle{headandfoot}

% Command
\newcommand{\doublebar}[0]{
    \noindent\rule{\textwidth}{3pt}
    \rule[3mm]{\textwidth}{0.5pt}
}

\newcounter{compteurpartie}
\newcounter{compteurexercice}

\newcommand{\partie}[1]{
    \stepcounter{compteurpartie}
    \section*{Partie \arabic{compteurpartie} : \hspace{1pt} #1}
    \setcounter{compteurexercice}{0} 
}

\newcommand{\exercice}[2]{
    \stepcounter{compteurexercice}
    \section*{Exercice \arabic{compteurexercice} \tiny{/#2 points} \hrulefill \hspace{1pt} #1}
}

\newcommand{\qcm}[5]{
    - #1
    \\
    \begin{itemize}
        \ifthenelse{\equal{#2}{}}{}{%
            \item[$\square$] #2
        }
        \ifthenelse{\equal{#3}{}}{}{%
            \item[$\square$] #3
        }
        \ifthenelse{\equal{#4}{}}{}{%
            \item[$\square$] #4
        }
        \ifthenelse{\equal{#5}{}}{}{%
            \item[$\square$] #5
        }
    \end{itemize}
    \vspace{2mm}
    \rule[3mm]{\textwidth}{0.5pt}
}

% Document
\begin{document}

% Header
\firstpageheader{Nom : \\ Prénom :}
  {}
  {Classe : \texttt{Première} \\ Date : \today}
\firstpagefooter{}{Page~\thepage}{}

\vspace{2em}
\begin{center}
    \textbf{Les dictionnaires}

    \vspace{1em}
    DS
\end{center}
    
\doublebar

\section*{Le barème}

\begin{tabular}{|c|c|c|c|}
    \hline
    \textbf{Exercice} & \textbf{Temps recommandé} & \textbf{Points} & \textbf{Points} \\ \hline
    1.1 & 10 minutes & 5 pts & ... pts\\
    \hline
    2.1 & 20 minutes & 5 pts & ... pts\\
    \hline
    2.2 & 30 minutes & 10 pts & ... pts\\
    \hline
\end{tabular}

% Content
\partie{Connaissances de cours}

\exercice{Questionnaire à choix multiples}{5}

\qcm{Laquelle de ces syntaxes est la bonne pour créer un dictionnaire ?}
{\texttt{dico = [a:12, b:24, c:35]}}
{\texttt{dico = \{a:12, b:24, c:35\}}}
{\texttt{dico = [cle1:valeur, cle:valeur, cle:valeur]}}
{}

\qcm{Un dictionnaire est toujours trié ?}
{\texttt{Vrai}}
{\texttt{Faux}}
{}{}

\qcm{Comment ajouter un nouvel élément à un dictionnaire existant ?}
{\texttt{dico.append('cle', 'valeur')}}
{\texttt{dico['cle'] = 'valeur'}}
{\texttt{dico.add('cle', 'valeur')}}
{}

\qcm{Comment supprimer une clé et sa valeur d'un dictionnaire ?}
{\texttt{del dico['cle']}}
{\texttt{dico.pop('cle')}}
{\texttt{dico.remove('cle')}}
{}


\qcm{Que retourne \texttt{'cle' in dico} si la clé existe dans le dictionnaire ?}
{\texttt{True}}
{\texttt{False}}
{}{}

\partie{Partie pratique}

\exercice{Notes}{5}
Faites un code python pour calculer la moyenne de Alain, Nathalie et Robert en allant chercher leurs notes dans le dictionnaire \texttt{carnet\_notes} ci dessous :

\begin{mintedbox}{python}
carnet_notes = {
    "Alain": [12, 15 , 17],
    "Nathalie" : [15, 13 , 16],
    "Robert": [13, 15 , 11]
}
\end{mintedbox}
\newpage

\exercice{Alphabet}{10}
Faites ci-dessous une fonction \texttt{compter\_lettres(texte)} qui prend en paramètre une chaine de caractère et renvoie un dictionnaire ayant une clé pour chaque lettre de l'alphabet, qui a pour valeur, le nombre d'occurence de cette lettre dans le texte.

\end{document}